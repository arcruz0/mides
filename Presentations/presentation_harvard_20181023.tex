\documentclass{beamer}
%%%%%%%%%%%%%%%%%%%%%%%%%%%%%%%%%%%%%%%%%%%%%%%%%%%%%%%%%%%%%%%%%%%%%%%%%%%%%%%%%%%%%%%%%%%%%%%%%%%%%%%%%%%%%%%%%%%%%%%%%%%%%%%%%%%%%%%%%%%%%%%%%%%%%%%%%%%%%%%%%%%%%%%%%%%%%%%%%%%%%%%%%%%%%%%%%%%%%%%%%%%%%%%%%%%%%%%%%%%%%%%%%%%%%%%%%%%%%%%%%%%%%%%%%%%%
\usepackage[utf8]{inputenc}
\usetheme{Madrid}
\usecolortheme{beaver}

\usepackage{eurosym}
\usepackage{graphicx}
\usepackage{amsmath}
\usepackage{amsfonts}
\usepackage{amssymb}
\usepackage{booktabs}
\usepackage{varioref}
\usepackage{setspace}
\usepackage{amsthm} 
\usepackage{lscape}
\usepackage{epstopdf}
\usepackage{ifthen}
\usepackage{setspace}
\usepackage{multicol}
\usepackage{hyperref}
\usepackage{graphicx}
\usepackage{tabularx}
\usepackage{booktabs}
\usepackage{verbatim}
\usepackage{longtable}
\usepackage{amssymb}
\usepackage{pdflscape}
\usepackage{palatino}
\usepackage{rotating}
\usepackage{changepage}
\usepackage{alltt}
\usepackage{parskip}
\usepackage{textcomp}
\usepackage{enumerate}
\usepackage{dcolumn}
\usepackage{adjustbox}
\usepackage{xcolor}
\usepackage[hang,flushmargin]{footmisc} 
\newcommand*{\myalign}[2]{\multicolumn{1}{#1}{#2}}
\setlength{\pdfpagewidth}{8.5in} \setlength{\pdfpageheight}{11in}
\newcolumntype{d}[1]{D{.}{.}{2.3} }

\newtheorem{proposition}[theorem]{Proposition}

\setcounter{MaxMatrixCols}{10}


\title[Retargeting cash transfer programs] %optional
{Retargeting cash transfer programs in Uruguay}

 
\author[Alejandro Lagomarsino] % (optional, for multiple authors)
{Alejandro Lagomarsino \& Lihuen Nocetto\inst{*} }

\institute[Harvard University] % (optional)
{
	\inst{*}%
	Harvard University\\
	Pontifical Catholic University of Chile
}

\date[October. 2018] % (optional)
{Development Lunch. October 23, 2018}
 
\graphicspath{ {../Empirical_analysis/Analysis/Output/} }	% Sets folder for \includegraphics

\begin{document}
	\setlength{\parindent}{10pt}
	
	\frame{\titlepage}

\begin{frame}
\frametitle{Motivation}
\begin{itemize}
	\item Mistargeting of cash transfer programs is ubiquitous in the developing world.
	\begin{itemize}
		\item Brazil: coverage = 55\%; leakage = 50\%
		\item Mexico: coverage = 53\%; leakage = 61\%
	\end{itemize}
	\item Retargeting could produce substantial financial gains in terms of poverty alleviation and fiscal savings (Robles, Rubio \& Stampini (2015))
	\begin{itemize}
		\item These type of arguments rest on static models with no loss aversion.
	\end{itemize}
\end{itemize}
\end{frame}

\begin{frame}
\frametitle{Research questions}
\begin{itemize}
	\item How do people respond to losses ands gains of governmental cash transfer?
	\item How peer effects exacerbate or attenuate the impact of gaining or losing a cash transfer?
	\item How can we use these results to assess the costs and benefits of retargeting cash transfer programs?	
\end{itemize}
\end{frame}

\begin{frame}
\frametitle{What this paper is about}
\begin{itemize}
\item FRDD in which households that were initially receiving 1 cash transfer, ``randomly'' kept it, lost it, or doubled it, and how that impacts:
\begin{itemize}
	\item Prosocial behavior
	\item Access to other government welfare and non-welfare programs
	\item Domestic violence
	\item Food consumption
	\item Durable goods
	\item Education
	\item Income/Employment
	\item Health
\end{itemize}
\end{itemize}
\end{frame}

\begin{frame}
\frametitle{Outline}
\begin{itemize}
\item Background
\item Data
\item Discussion of the empirical strategy
\item Descriptive binscatters
\end{itemize}
\end{frame}

\begin{frame}
\frametitle{Uruguay}
\begin{itemize}
\item \textbf{Population}: 3.5 MM
\item \textbf{GDP Per Capita USD}: 16,000 (2017)
\item \textbf{Avg. annual real GDP growth 2011-2017}: 3.0 \%
\item \textbf{Poverty, less than \$5.5 (2011 PPP)}: 2011: 6\%; 2016: 3.7\%	
\end{itemize}
\begin{center}
	\includegraphics[width=60mm]{map.png}
	\label{map}
\end{center}
\end{frame}

\begin{frame}
\frametitle{UCT program: food card}
\begin{itemize}
\item \textbf{Managed by}: Ministry of Social Development.
\item \textbf{Amount}: Depends in a non-linear way in number of children. 47 USD (monthly) in 2012 for household with two kids. Double amount for poorest 30,000 households.
\item \textbf{Beginnings}: PANES households with children or pregnant women administratively enrolled in 2008 + 20,000 INDA households in 2009.
\item \textbf{Target}: Initially ``all households with children in extreme poverty''.		
\end{itemize}
\end{frame}

\begin{frame}
\frametitle{Targeting issues}
\begin{itemize}
\item \textbf{Recognized}: On October 2011 a report by the Ministry identified severe targeting issues.
\begin{itemize}
\item Households in extreme poverty were 10,000 in 2009 and beneficiaries were 85,000.
\item Not clear which is the target population: 5 scenarios considered for UCT.
\item Type I error for UCT: 37\%, 39\%, 63\%, 42\%, 43\%.
\item Type II error for UCT: 92\%, 82\%, 29\%, 64\%, 34\%.
\end{itemize}
\item \textbf{Steps taken}:
\begin{itemize}
\item Set target population to 60,000 poorest households.
\item Proxy means-tested approach to decide whether to grant or withdraw UCT benefits.
\item Increase household visits to (re) assess their status in the program.
\end{itemize}	
\end{itemize}
\end{frame}

\begin{frame}
\frametitle{Number of household visits}
\begin{center}
\includegraphics[width=105mm]{nHouse.png}
\label{nHouse}
\end{center}
\end{frame}

\begin{frame}
\frametitle{Questionnaire}
\begin{center}
\includegraphics[width=105mm]{mides.jpg}
\label{mides}
\end{center}
\end{frame}

\begin{frame}
\frametitle{Vulnerability Index}
\begin{itemize}
\item Probit where $Y$ is a dummy equal to 1 when household belongs to the first income quantile.
\item Model run separately for Montevideo and the rest of the country.
\item Thresholds defined to capture target population.	
\end{itemize}
\end{frame}

\begin{frame}
\frametitle{Data: sources and variables}
\begin{itemize}
	\item Ministry of Development: Universe of household visits conducted by the Ministry from January 2011 until July 2018 {\color{gray}(georeferenced)}.
	\item Social Security Bank: Monthly data on formal income, employment, CCT for visited-CCT recipients.
	\item Montevideo's municipal government: Individual level data on turnout during the 2008, 2011, 2013 and 2016 Participatory Budgeting elections.	
	\item {\color{gray}SIIAS: access to government welfare and non-welfare programs, education, health, employment.}
	\item {\color{gray}Data we are requesting: vandalism, consumer credit, other electoral turnout data, participation in political parties, water and electricity claims.}
\end{itemize}
\end{frame}

\begin{frame}
\frametitle{Data: our sample}
\begin{center}
\includegraphics[width=115mm]{summ.png}
\label{summ}
\end{center}
\end{frame}


\begin{frame}
\frametitle{Empirical strategy}
\begin{itemize}
	\item Compare households that lost/gained a transfer x months before/after the event. 
	\item FRDD with VI as running variable and two thresholds.
\end{itemize}
\end{frame}


\begin{frame}
\frametitle{RDD with overlapping control functions}
\begin{center}
\includegraphics[width=105mm]{pic1.png}
\label{pic1}
\end{center}
\end{frame}


\begin{frame}
\frametitle{RDD with overlapping control functions}
\begin{center}
\includegraphics[width=105mm]{pic2.png}
\label{pic2}
\end{center}
\end{frame}

\begin{frame}
\frametitle{RDD with overlapping control functions}
\begin{center}
\includegraphics[width=105mm]{pic3.png}
\label{pic3}
\end{center}
\end{frame}

\begin{frame}
\frametitle{RDD with overlapping control functions}
\begin{align*}
Y_{i} &= {\color{blue}{\beta _0\mathbf{1}[VI_{i}<K]}} + \beta_1 \mathbf{1}[VI_{i}>T_{1}]\mathbf{1}[VI_{i}<K]\\	   
&+{\color{blue}{\beta_2 (VI_{i}-T_{1})\mathbf{1}[VI_{i}<K] + \beta_3 (VI_{i}-T_{1})\mathbf{1}[VI_{i}>T_{1}]\mathbf{1}[VI_{i}<K]}} \\
&+ {\color{green}{\gamma_0 (1-\mathbf{1}[VI_{i}<K])}} + \gamma_1 \mathbf{1}[VI_{i}>T_{2}]  \\
&+ {\color{green}{\gamma_2 (VI_{i}-T_{2})(1-\mathbf{1}[VI_{i}<K]) + \gamma_3 (VI_{i}-T_{2})\mathbf{1}[VI_{i}>T_{2}]}} + \epsilon_i
\end{align*}
With $K=\frac{T_{1} + T_{2}}{2}$ and restrictions: $\gamma_2 = \beta_3$ , $ \gamma_0 = \beta_0 + \beta_3 K$.\\
\end{frame}

\begin{frame}
\frametitle{The stable unit treatment value assumption}
\begin{center}
\includegraphics[width=105mm]{visits_area.jpg}
\label{visits_area}
\end{center}
\end{frame}

\begin{frame}
\frametitle{The stable unit treatment value assumption}
\begin{center}
\includegraphics[width=105mm]{visits.jpg}
\label{visits}
\end{center}
\end{frame}

\begin{frame}
\frametitle{The stable unit treatment value assumption}
Voting turnout in 2013 for 16+ individuals that were living in households ``area'' visited in the September 2011 - March 2013 period:
\begin{multline}
Votes_{i,h,a,2013} = \beta _0 + \beta_1ExtremeAdj_{a,2012} +\beta_2CloseAdj_{a,2012} + \\
\beta_3\mathbf{1}[VI_{h,a,2012}>0] + f(VI_{h,a,2012}) + \Phi X_{i,h,a,2012} + \epsilon_{i,h,a,2013}
\end{multline}
\begin{itemize}
\item Leakage is determinant of increase in crime and decrease in participation in community groups, while under-coverage is not (Cameron \& Shah (2014))
\item Receiving the CCT: strong impact on trust in MIDES, decreases participation in community groups, mild negative impact on interpersonal trust (Bergolo et al. (2015))
\item Proxy-means testing may seem arbitrary (Coady et al. (2004))
\end{itemize}
\end{frame}	

\begin{frame}
\frametitle{Distribution of visits by Vulnerability Index}
\begin{center}
\includegraphics[width=61mm]{mdeodistrib.png}
\includegraphics[width=61mm]{intdistrib.png}
\label{dist}
\end{center}
\end{frame}

\begin{frame}
\frametitle{McCrary test}
\begin{center}
	\includegraphics[width=50mm]{mccraryPrimTot.png} 
	\includegraphics[width=50mm]{mccrarySecTot.png}
	\label{mccrary}
\end{center}
\begin{itemize}
	\item UCT first threshold t-stat: 0.16
	\item UCT second threshold t-stat: 4.39
	\item CCT threshold t-stat: 0.75
\end{itemize}
\end{frame}

\begin{frame}
\frametitle{First stage for non-beneficiaries: 1 month after the visit}
\begin{center}
\includegraphics[width=105mm]{int_noTus_tus1.png}
\label{int_noTus_tus1}
\end{center}
\end{frame}

\begin{frame}
\frametitle{First stage for non-beneficiaries: 3 month after the visit}
\begin{center}
	\includegraphics[width=105mm]{int_noTus_tus3.png}
	\label{int_noTus_tus3}
\end{center}
\end{frame}

\begin{frame}
\frametitle{First stage for non-beneficiaries: 6 month after the visit}
\begin{center}
\includegraphics[width=105mm]{int_noTus_tus6.png}
\label{int_noTus_tus6}
\end{center}
\end{frame}

\begin{frame}
\frametitle{First stage for non-beneficiaries: 9 month after the visit}
\begin{center}
\includegraphics[width=105mm]{int_noTus_tus9.png}
\label{int_noTus_tus9}
\end{center}
\end{frame}

\begin{frame}
\frametitle{First stage for non-beneficiaries: 1 year after the visit}
\begin{center}
\includegraphics[width=105mm]{int_noTus_tus12.png}
\label{int_noTus_tus12}
\end{center}
\end{frame}

\begin{frame}
\frametitle{No food because of lack of money in the previous 30 days}
\begin{center}
	\includegraphics[width=105mm]{sinalimentos.png}
	\label{sinalimentos}
\end{center}
\end{frame}

\begin{frame}
\frametitle{No food for adults}
\begin{center}
\includegraphics[width=105mm]{adultonocomio.png}
\label{adultonocomio}
\end{center}
\end{frame}

\begin{frame}
\frametitle{No food for minors}
\begin{center}
\includegraphics[width=105mm]{menornocomio.png}
\label{menornocomio}
\end{center}
\end{frame}

\begin{frame}
\frametitle{Attends school (18 years old or less)}
\begin{center}
\includegraphics[width=105mm]{asisteEscuela.png}
\label{asisteEscuela}
\end{center}
\end{frame}

\begin{frame}
\begin{center}
	{\Huge Questions?\par}
\end{center}
\end{frame}

\begin{frame}
\begin{center}
	{\Huge Thank you\par}
\end{center}
\end{frame}

\end{document}	