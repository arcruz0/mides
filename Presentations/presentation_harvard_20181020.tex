\documentclass{beamer}
%%%%%%%%%%%%%%%%%%%%%%%%%%%%%%%%%%%%%%%%%%%%%%%%%%%%%%%%%%%%%%%%%%%%%%%%%%%%%%%%%%%%%%%%%%%%%%%%%%%%%%%%%%%%%%%%%%%%%%%%%%%%%%%%%%%%%%%%%%%%%%%%%%%%%%%%%%%%%%%%%%%%%%%%%%%%%%%%%%%%%%%%%%%%%%%%%%%%%%%%%%%%%%%%%%%%%%%%%%%%%%%%%%%%%%%%%%%%%%%%%%%%%%%%%%%%
\usepackage[utf8]{inputenc}
\usetheme{Madrid}
\usecolortheme{beaver}

\usepackage{eurosym}
\usepackage{graphicx}
\usepackage{amsmath}
\usepackage{amsfonts}
\usepackage{amssymb}
\usepackage{booktabs}
\usepackage{varioref}
\usepackage{setspace}
\usepackage{amsthm} 
\usepackage{lscape}
\usepackage{epstopdf}
\usepackage{ifthen}
\usepackage{setspace}
\usepackage{multicol}
\usepackage{hyperref}
\usepackage{graphicx}
\usepackage{tabularx}
\usepackage{booktabs}
\usepackage{verbatim}
\usepackage{longtable}
\usepackage{amssymb}
\usepackage{pdflscape}
\usepackage{palatino}
\usepackage{rotating}
\usepackage{changepage}
\usepackage{alltt}
\usepackage{parskip}
\usepackage{textcomp}
\usepackage{enumerate}
\usepackage{dcolumn}
\usepackage{adjustbox}
\usepackage{xcolor}
\usepackage[hang,flushmargin]{footmisc} 
\newcommand*{\myalign}[2]{\multicolumn{1}{#1}{#2}}
\setlength{\pdfpagewidth}{8.5in} \setlength{\pdfpageheight}{11in}
\newcolumntype{d}[1]{D{.}{.}{2.3} }

\newtheorem{proposition}[theorem]{Proposition}

\setcounter{MaxMatrixCols}{10}


\title[Retargeting cash transfer programs] %optional
{Retargeting cash transfer programs in Uruguay}

 
\author[Alejandro Lagomarsino] % (optional, for multiple authors)
{Alejandro Lagomarsino \& Lihuen Nocetto\inst{*} }

\institute[Harvard University] % (optional)
{
	\inst{*}%
	Harvard University\\
	Pontifical Catholic University of Chile
}

\date[October. 2018] % (optional)
{Development Lunch. October 23, 2018}
 
\graphicspath{ {../Empirical_analysis/Analysis/Output/} }	% Sets folder for \includegraphics

\begin{document}
	\setlength{\parindent}{10pt}
	
	\frame{\titlepage}

\begin{frame}
\frametitle{Motivation}
\begin{itemize}
	\item Mistargeting of cash transfer programs is ubiquitous in the developing world.
	\item Retargeting could produce susbstantial financial gains in terms of poverty alleviation and fiscal savings (Robles, Rubio \& Stampini (2015))
	\begin{itemize}
		\item Logic behind the argument lies on a static model with no loss aversion.
	\end{itemize}
\end{itemize}
\end{frame}

\begin{frame}
\frametitle{Research questions}
\begin{itemize}
	\item Behavior differs when experiencing a loss relative to a gain of a governmental cash transfer?
	\item How can we use these results to assess the costs and benefits of retargeting cash transfer programs?
	\item How peer effects exacerbate or attenuate the impact of gaining or losing a cash transfer?
\end{itemize}
\end{frame}

\begin{frame}
\frametitle{What this paper is about}
\begin{itemize}
	\item FRDD in which households that were initially receiving 1 cash transfer, ``randomly'' kept it, lost it, or doubled it, and how that impacts:
	\begin{itemize}
	\item Pro-social behavior
	\item Food consumption
	\item Durable goods consumption
	\item Access to other government welfare and non-welfare programs 
	\item Education
	\item Income/Employment:
	\item Health
	\item Domestic violence
	\end{itemize}
\end{itemize}
\end{frame}

\begin{frame}
\frametitle{Outline}
\begin{itemize}
	\item Background
	\item Data
	\item Proxy means test
	\item Empirical strategies
	\item Results (very few)
	\item Further steps
\end{itemize}
\end{frame}

\begin{frame}
\frametitle{Uruguay}
\begin{itemize}
	\item \textbf{Population}: 3.5 MM
	\item \textbf{GDP Per Capita USD}: 16,000 (2017)
	\item \textbf{Avg. annual real GDP growth 2011-2017}: 3.0 \%
	\item \textbf{Poverty, less than \$5.5 (2011 PPP)}: 2011: 6\%; 2016: 3.7\%	
\end{itemize}
\end{frame}

\begin{frame}
\frametitle{UCT program: food card}
\begin{itemize}
	\item \textbf{Beginnings}: PANES households with children or pregnant women administratively enrolled in 2008 + 20,000 INDA households in 2009.
	\item \textbf{Target}: Initially ``all households with children in extreme poverty''.
	\item \textbf{Amount}: Depends in a non-linear way in number of children. 47 USD in 2012 for household with two kids, one in primary school, one in secondary (13\% monthly minimum wage). Double amount for poorest 30,000 households (January 2011).
	\item \textbf{Managed by}: Ministry of Social Development.		
\end{itemize}
\end{frame}
	
\begin{frame}
\frametitle{Targeting issues}
		\begin{itemize}
			\item \textbf{Recognized}: On October 2011 a report by the Ministry identified severe targeting issues.
			\begin{itemize}
				\item Households in extreme poverty were 10,000 in 2009 and beneficiaries were 85,000.
				\item Not clear which is the target population: 5 scenarios considered for UCT.
				\item Type I error for UCT: 37\%, 39\%, 63\%, 42\%, 43\%.
				\item Type II error for UCT: 92\%, 82\%, 29\%, 64\%, 34\%.
			\end{itemize}
			\item \textbf{Steps taken}:
			\begin{itemize}
				\item Set target population to 60,000 poorest households.
				\item Proxy means-tested approach to decide whether to grant or withdraw UCT benefits.
				\item Increase household visits: 6,000 per year in 2008-2011 and 81,178 from September 2011 to July 2013; 14,000 in 2017.
			\end{itemize}	
		\end{itemize}
\end{frame}

\begin{frame}
\frametitle{Number of household visits}
\begin{center}
	\includegraphics[width=105mm]{nHouse.png}
	\label{nHouse}
\end{center}
\end{frame}


\begin{frame}
\frametitle{Questionnaire}
\begin{center}
	\includegraphics[width=105mm]{mides.jpg}
	\label{mides}
\end{center}
\end{frame}

\begin{frame}
\frametitle{Vulnerability Index}
\begin{itemize}
	\item Probit where $Y$ is a dummy equal to 1 when household belongs to the first income quantile.
	\item Model run separately for Montevideo and the rest of the country.
	\item Thresholds defined to capture target population.	
\end{itemize}
\end{frame}

\begin{frame}
\frametitle{Data}
\begin{itemize}
	\item Ministry of Development: Universe of household visits performed by the Ministry from January 2011 until July 2018 {\color{gray}(georeferenced)}.
	\item Montevideo's municipal government: Individual level data on turnout during the 2008, 2011, 2013 and 2016 Participatory Budgeting elections.
	\item {\color{gray}Social Security Bank: Monthly data on formal income, employment, CCT for visited-CCT recipients.}	
	\item {\color{gray}SIIAS: access to government welfare and non-welfare programs, education, health, employment.}
	\item {\color{gray}Data we are requesting: vandalism, consumer credit, other electoral turnout data, participation in political parties, water and electricity claims.}
\end{itemize}
\end{frame}

\begin{frame}
\frametitle{Data}
\begin{center}
	\includegraphics[width=105mm]{summ.png}
	\label{summ}
\end{center}
\end{frame}

\begin{frame}
\frametitle{Distribution of visits by Vulnerability Index}
\begin{center}
	\includegraphics[width=61mm]{mdeodistrib.png}
	\includegraphics[width=61mm]{intdistrib.png}
	\label{dist}
\end{center}
\end{frame}

\begin{frame}
\frametitle{McCrary test}
	\begin{center}
		\includegraphics[width=50mm]{mccraryPrimTot.png} 
		\includegraphics[width=50mm]{mccrarySecTot.png}
		\label{mccrary}
	\end{center}
	\begin{itemize}
		\item UCT first threshold t-stat: 0.16
		\item UCT second threshold t-stat: 4.39
		\item CCT threshold t-stat: 0.75
	\end{itemize}
\end{frame}

\begin{frame}
\frametitle{First stage for non-beneficiaries: 1 month after the visit}
\begin{center}
	\includegraphics[width=105mm]{int_noTus_tus1.png}
	\label{int_noTus_tus1}
\end{center}
\end{frame}

\begin{frame}
\frametitle{First stage for non-beneficiaries: 3 month after the visit}
\begin{center}
	\includegraphics[width=105mm]{int_noTus_tus3.png}
	\label{int_noTus_tus3}
\end{center}
\end{frame}

\begin{frame}
\frametitle{First stage for non-beneficiaries: 6 month after the visit}
\begin{center}
	\includegraphics[width=105mm]{int_noTus_tus6.png}
	\label{int_noTus_tus6}
\end{center}
\end{frame}

\begin{frame}
\frametitle{First stage for non-beneficiaries: 9 month after the visit}
\begin{center}
	\includegraphics[width=105mm]{int_noTus_tus9.png}
	\label{int_noTus_tus9}
\end{center}
\end{frame}

\begin{frame}
\frametitle{First stage for non-beneficiaries: 1 year after the visit}
\begin{center}
	\includegraphics[width=105mm]{int_noTus_tus12.png}
	\label{int_noTus_tus12}
\end{center}
\end{frame}

\begin{frame}
\frametitle{Empirical strategy: Impact of gaining or losing a transfer}
\begin{itemize}
	\item FRDD with VI as running variable.
	\item FRDD with number of months before/after the event that a household lost/gained the transfer.
	\item Diff-in-Diff
\end{itemize}
\end{frame}

\begin{frame}
\frametitle{RDD with overlapping control functions}
\begin{center}
	\includegraphics[width=105mm]{pic1.png}
	\label{pic1}
\end{center}
\end{frame}

\begin{frame}
\frametitle{RDD with overlapping control functions}
\begin{center}
	\includegraphics[width=105mm]{pic2.png}
	\label{pic2}
\end{center}
\end{frame}

\begin{frame}
\frametitle{RDD with overlapping control functions}
\begin{center}
	\includegraphics[width=105mm]{pic3.png}
	\label{pic3}
\end{center}
\end{frame}

\begin{frame}
\frametitle{RDD with overlapping control functions}
\begin{multline}
Y_{i} = \beta _0 + \beta_1 \mathbf{1}[VI_{i}>T_{1}]*{1}[VI_{i}<\frac{T_{1}+T{2}{2}] +\beta_2CloseAdj_{a,2012} + \\
\beta_3\mathbf{1}[VI_{h,a,2012}>0] + f(VI_{h,a,2012}) + \Phi X_{i,h,a,2012} + \epsilon_{i,h,a,2013}
\end{multline}
\end{frame}

\begin{frame}
\frametitle{The stable unit treatment value assumption}
\begin{center}
	\includegraphics[width=105mm]{visits_area.jpg}
	\label{visits_area}
\end{center}
\end{frame}

\begin{frame}
\frametitle{The stable unit treatment value assumption}
\begin{center}
	\includegraphics[width=105mm]{visits.jpg}
	\label{visits}
\end{center}
\end{frame}
	
\begin{frame}
\frametitle{Results: No food because of lack of money in the previous 30 days}
\begin{itemize}
	\item Ideal experiment: from a pool of people initially receiving a transfer, randomly duplicate the transfer amount to part of the group and substract it from the rest.
	\item Closest we can get in our setting: Difference-in-Difference
	\begin{itemize}
		\item 27,543 were receiving a single UCT by September 2011 and were visited in the period September 2011 - July 2013.
		\item 15,278 stopped receiving the transfer.
		\item 7,329 duplicated their amount.
	\end{itemize}
	\item Other approach: two separate RDs.
\end{itemize}
\end{frame}

\begin{frame}
\frametitle{Potential contributions}
\begin{itemize}
	\item First paper to look at impact of losing a cash transfer from a CCT or UCT program.
	Intra-household: heterogeneous effects across members of the household 
	\item Length: heterogeneous effects across ``old'' and ``recent'' beneficiaries.
	\item Motive matters: Losing the transfer due to VI, educational conditionality or age (DID with multiple treatments, see Fricke (2017)).
\end{itemize}
\end{frame}
	
\end{document}	