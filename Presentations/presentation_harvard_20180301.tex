\documentclass{beamer}
%%%%%%%%%%%%%%%%%%%%%%%%%%%%%%%%%%%%%%%%%%%%%%%%%%%%%%%%%%%%%%%%%%%%%%%%%%%%%%%%%%%%%%%%%%%%%%%%%%%%%%%%%%%%%%%%%%%%%%%%%%%%%%%%%%%%%%%%%%%%%%%%%%%%%%%%%%%%%%%%%%%%%%%%%%%%%%%%%%%%%%%%%%%%%%%%%%%%%%%%%%%%%%%%%%%%%%%%%%%%%%%%%%%%%%%%%%%%%%%%%%%%%%%%%%%%
\usepackage[utf8]{inputenc}
\usetheme{Madrid}
\usecolortheme{beaver}

\usepackage{eurosym}
\usepackage{graphicx}
\usepackage{amsmath}
\usepackage{amsfonts}
\usepackage{amssymb}
\usepackage{booktabs}
\usepackage{varioref}
\usepackage{setspace}
\usepackage{amsthm} 
\usepackage{lscape}
\usepackage{epstopdf}
\usepackage{ifthen}
\usepackage{setspace}
\usepackage{multicol}
\usepackage{hyperref}
\usepackage{graphicx}
\usepackage{tabularx}
\usepackage{booktabs}
\usepackage{verbatim}
\usepackage{longtable}
\usepackage{amssymb}
\usepackage{pdflscape}
\usepackage{palatino}
\usepackage{rotating}
\usepackage{changepage}
\usepackage{alltt}
\usepackage{parskip}
\usepackage{textcomp}
\usepackage{enumerate}
\usepackage{dcolumn}
\usepackage{adjustbox}
\usepackage[hang,flushmargin]{footmisc} 
\newcommand*{\myalign}[2]{\multicolumn{1}{#1}{#2}}
\setlength{\pdfpagewidth}{8.5in} \setlength{\pdfpageheight}{11in}
\newcolumntype{d}[1]{D{.}{.}{2.3} }

\newtheorem{proposition}[theorem]{Proposition}

\setcounter{MaxMatrixCols}{10}


\title[Retargeting cash transfers programs] %optional
{Retargeting cash transfer programs in Uruguay}

 
\author[Alejandro Lagomarsino] % (optional, for multiple authors)
{Alejandro Lagomarsino \& Lihuen Nocetto\inst{*} }

\institute[Harvard University] % (optional)
{
	\inst{*}%
	Harvard University\\
	Pontifical Catholic University of Chile
}

\date[March. 2018] % (optional)
{PE Lunch. March 1, 2018}
 
 
\begin{document}
	\setlength{\parindent}{10pt}
	
	\frame{\titlepage}

\begin{frame}
		\frametitle{Research question}
		\begin{itemize}
			\item How do beneficiaries of conditional and unconditional cash transfer programs (CCT and UCT) react when the government stops transferring them money?
			\item 5 dimensions:
			\begin{itemize}
				\item Citizen's participation in local public good provision
				\item Quality of the public space
				\item Access to other governmental welfare and non-welfare programs
				\item Schooling
				\item Labor force participation
			\end{itemize}
		\end{itemize}
\end{frame}

\begin{frame}
		\frametitle{CCT program: children's allowance}
		\begin{itemize}
			\item \textbf{Beginnings}: Proxy means-tested, automatic applications from PANES beneficiaries in January 2008; regular applications opened in April 2008.
			\item \textbf{Target}: 500,000 children in worst socioeconomic conditions (402,898 in 2012).
			\item \textbf{Amount}: Depends in a non-linear way in number of children and schooling. 90 USD in 2012 for household with two kids, one in primary school, one in secondary (25\% monthly minimum wage).
			\item \textbf{Conditionality}: School attendance (enforcement started in 2014) and health checks (not enforced).
			\item \textbf{Managed by}: Social Security Bank.	
		\end{itemize}
\end{frame}

\begin{frame}
\frametitle{Assignment to CCT (Bergolo \& Galvan (2016))}
\begin{center}
	\includegraphics[width=105mm]{afam.jpg}
	\label{afam}
\end{center}
\end{frame}

\begin{frame}
\frametitle{UCT program: food card}
\begin{itemize}
	\item \textbf{Beginnings}: PANES households with children or pregnant women administratively enrolled in 2008 + 20,000 INDA households in 2009.
	\item \textbf{Target}: Initially ``all households with children in extreme poverty''.
	\item \textbf{Amount}: Depends in a non-linear way in number of children. 47 USD in 2012 for household with two kids, one in primary school, one in secondary (13\% monthly minimum wage). Double amount for poorest 30,000 households (January 2011).
	\item \textbf{Managed by}: Ministry of Social Development.		
\end{itemize}
\end{frame}
	
\begin{frame}
\frametitle{Targeting issues}
		\begin{itemize}
			\item \textbf{Recognized}: On October 2011 a report by the Ministry identified severe targeting issues.
			\begin{itemize}
				\item Households in extreme poverty were 10,000 in 2009 and beneficiaries were 85,000.
				\item Not clear which is the target population: 5 scenarios considered for UCT.
				\item Type I error for UCT: 37\%, 39\%, 63\%, 42\%, 43\%.
				\item Type II error for UCT: 92\%, 82\%, 29\%, 64\%, 34\%.
			\end{itemize}
			\item \textbf{Steps taken}:
			\begin{itemize}
				\item Set target population to 60,000 poorest households.
				\item Proxy means-tested approach to decide whether to grant or withdraw UCT benefits.
				\item Increase household visits: 6,000 per year in 2008-2011 and 81,178 from September 2011 to July 2013; 14,000 in 2017.
			\end{itemize}	
		\end{itemize}
\end{frame}

\begin{frame}
\frametitle{Visits}
\begin{center}
	\includegraphics[width=105mm]{mides.jpg}
	\label{mides}
\end{center}
\end{frame}

\begin{frame}
\frametitle{UCT (DINEM (2012))}
\begin{center}
	\includegraphics[width=105mm]{tus.jpg}
	\label{tus}
\end{center}
\end{frame}

\begin{frame}
\frametitle{Outcome of the 81,178 visits}
\begin{center}
	\includegraphics[width=105mm]{visits_outcome.jpg}
	\label{tus}
\end{center}
\end{frame}

\begin{frame}
\frametitle{Regression discontinuity design}
Within sample of individuals that were living in a household getting single UCT before the visit and with VI in the threshold Single-None UCT (analogous for CCT):
\begin{multline}
Y_{i,h,t+n} = \beta _0 + \beta_1\mathbf{1}[VI_{h,t}>0] + f(VI_{h,t}) + \Phi X_{i,h,t} + \epsilon_{i,h,t+n} 
\end{multline}
\end{frame}

\begin{frame}
\frametitle{Participatory Budgeting in Montevideo}
\begin{itemize}
	\item When? 2006, 2007, 2008, 2011, 2013 (October), 2016.
	\item Who can vote? 16+, residents in one of the 8 sub-municipalities in Montevideo.
	\item Turnout? Low, approximately 10\%.
	\item What? 910 proposals in 2013; 57 elected ($\approx$ USD 5 MM).
\end{itemize}
\end{frame}

\begin{frame}
\frametitle{Participatory Budgeting in Montevideo}
\includegraphics[width=50mm]{win_1.jpg}
\includegraphics[width=50mm]{win_2.jpg}
\includegraphics[width=50mm]{win_3.jpg}
\includegraphics[width=50mm]{win_4.jpg}
\end{frame}	

\begin{frame}
\frametitle{Outcomes}
\begin{itemize}
	\item \textbf{Gov programs}: housing programs, labor training program, \textit{nini's} support program, pensions, other food assistance programs (PANRN and INDA), day care (0-12).
	\item \textbf{Schooling}: enrollment, grade, attendance, reasons for not finishing school. 	 
	\item \textbf{Employment}: employment dummy, unemployment/maternity/ disease subsidy. Self-reported working status and income.
	\item \textbf{Public space}: garbage in your block, contaminated water in your block, vandalism.
	\item \textbf{Other}: health, domestic violence, adults/children not eating, drug abuse, illegal electricity connection.
\end{itemize}
\end{frame}

\begin{frame}
\frametitle{The stable unit treatment value assumption}
\begin{center}
	\includegraphics[width=105mm]{visits_area.jpg}
	\label{visits_area}
\end{center}
\end{frame}

\begin{frame}
\frametitle{The stable unit treatment value assumption}
\begin{center}
	\includegraphics[width=105mm]{visits.jpg}
	\label{visits}
\end{center}
\end{frame}

\begin{frame}
\frametitle{The stable unit treatment value assumption}
Voting turnout in 2013 for 18+ individuals that were living in households ``area'' visited in the September 2011 - March 2013 period:
\begin{multline}
Votes_{i,h,a,2013} = \beta _0 + \beta_1ExtremeAdj_{a,2012} +\beta_2CloseAdj_{a,2012} + \\
\beta_3\mathbf{1}[VI_{h,a,2012}>0] + f(VI_{h,a,2012}) + \Phi X_{i,h,a,2012} + \epsilon_{i,h,a,2013}
\end{multline}
\begin{itemize}
	\item Leakage is determinant of increase in crime and decrease in participation in community groups, while under-coverage is not (Cameron \& Shah (2014))
	\item Receiving the CCT: strong impact on trust in MIDES, decreases participation in community groups, mild negative impact on interpersonal trust (Bergolo et al. (2015))
	\item Proxy-means testing may seem arbitrary (Coady et al. (2004))
\end{itemize}

\end{frame}	
	
\begin{frame}
		\frametitle{Testing for asymmetries}
		\begin{itemize}
			\item Ideal experiment: from a pool of people initially receiving a transfer, randomly duplicate the transfer amount to part of the group and substract it from the rest.
			\item Closest we can get in our setting: Difference-in-Difference
			\begin{itemize}
				\item 27,543 were receiving a single UCT by September 2011 and were visited in the period September 2011 - July 2013.
				\item 15,278 stopped receiving the transfer.
				\item 7,329 duplicated their amount.
			\end{itemize}
			\item Other approach: two separate RDs.
		\end{itemize}
\end{frame}

\begin{frame}
\frametitle{Other results to explore}
\begin{itemize}
	\item Intra-household: heterogeneous effects across members of the household 
	\item Length: heterogeneous effects across ``old'' and ``recent'' beneficiaries.
	\item Motive matters: Losing the transfer due to VI, educational conditionality or age (DID with multiple treatments, see Fricke (2017)).
\end{itemize}
\end{frame}
	
\end{document}	