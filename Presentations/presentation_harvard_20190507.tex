\documentclass{beamer}
%%%%%%%%%%%%%%%%%%%%%%%%%%%%%%%%%%%%%%%%%%%%%%%%%%%%%%%%%%%%%%%%%%%%%%%%%%%%%%%%%%%%%%%%%%%%%%%%%%%%%%%%%%%%%%%%%%%%%%%%%%%%%%%%%%%%%%%%%%%%%%%%%%%%%%%%%%%%%%%%%%%%%%%%%%%%%%%%%%%%%%%%%%%%%%%%%%%%%%%%%%%%%%%%%%%%%%%%%%%%%%%%%%%%%%%%%%%%%%%%%%%%%%%%%%%%
\usepackage[utf8]{inputenc}
\usetheme{Madrid}
\usecolortheme{beaver}

\usepackage{eurosym}
\usepackage{graphicx}
\usepackage{amsmath}
\usepackage{amsfonts}
\usepackage{amssymb}
\usepackage{booktabs}
\usepackage{varioref}
\usepackage{setspace}
\usepackage{amsthm} 
\usepackage{lscape}
\usepackage{epstopdf}
\usepackage{ifthen}
\usepackage{setspace}
\usepackage{multicol}
\usepackage{hyperref}
\usepackage{graphicx}
\usepackage{tabularx}
\usepackage{booktabs}
\usepackage{verbatim}
\usepackage{longtable}
\usepackage{amssymb}
\usepackage{pdflscape}
\usepackage{palatino}
\usepackage{rotating}
\usepackage{changepage}
\usepackage{alltt}
\usepackage{parskip}
\usepackage{textcomp}
\usepackage{enumerate}
\usepackage{dcolumn}
\usepackage{adjustbox}
\usepackage{xcolor}
\usepackage[hang,flushmargin]{footmisc} 
\newcommand*{\myalign}[2]{\multicolumn{1}{#1}{#2}}
\setlength{\pdfpagewidth}{8.5in} \setlength{\pdfpageheight}{11in}
\newcolumntype{d}[1]{D{.}{.}{2.3} }

\newtheorem{proposition}[theorem]{Proposition}

\setcounter{MaxMatrixCols}{10}
\usenavigationsymbolstemplate{} % Takes out beamer navigation buttons
\setbeamertemplate{page number in head/foot}{} % Takes out numbering of slides

\title[The Welfare Ratchet Effect] %optional
{The Welfare Ratchet Effect: Evidence from Uruguay}

 
\author[Alejandro Lagomarsino] % (optional, for multiple authors)
{Alejandro Lagomarsino \& Lihuen Nocetto\inst{*} }

\institute[] % (optional)
{
	\inst{*}%
	Harvard University\\
	Pontifical Catholic University of Chile
}

\date[PF/Labor Lunch - May 2019] % (optional)
{PF/Labor Lunch. May 7, 2019}
 
\graphicspath{ {../Empirical_analysis/Analysis/Output/} }	% Sets folder for \includegraphics

\begin{document}
	\setlength{\parindent}{10pt}
	
	\frame{\titlepage}

\begin{frame}
\frametitle{Motivation}
\begin{itemize}
	\item Ratchet effect (Wagner 1863; Bird 1971).
	\item D'Addio (2015) finds evidence of upward ratcheting on social spending: ``increases in recessions are not matched by equivalent reductions in expansions leading to increasing social spending over time''.
	\item Political unpopularity of cutbacks may be behind this effect. But is that the whole story?
\end{itemize}
\end{frame}

\begin{frame}
\frametitle{Research topic}
\begin{itemize}
	\item Why could it be hard to scale back welfare programs and reduce social spending? We study the behavioral responses of welfare recipients when they lose a government cash transfer.
	\item How? Uruguayan government started in 2012 to use an algorithm to define eligibility to a food card program that had pre-existing (and ``not so poor'') beneficiaries. We follow beneficiaries that quasi-randomly lost (gained) the food card with detailed administrative data from several sources after the event happened.
\end{itemize}
\end{frame}

\begin{frame}
\frametitle{What this paper is about in a nutshell}
\begin{itemize}
\item DID and FRDD in which individuals that were receiving one (zero) cash transfer, quasi-randomly lost (gained) it, and how that impacts:
\begin{itemize}
	\item Formal labor supply
	\item Enrollment in other welfare programs
	\item Complaints to the government
	\item Food insecurity
\end{itemize}
\item First paper to look at losses of UCTs or CCTs.
\end{itemize}
\end{frame}

\begin{frame}
\frametitle{Outline}
\begin{itemize}
\item Background
\item Empirical strategy
\item First stage
\item Results
\item Discussion on robustness checks
\item Conclusions and further steps
\end{itemize}
\end{frame}

\begin{frame}
\frametitle{Uruguay}
\begin{itemize}
\item \textbf{Population}: 3.5 MM
\item \textbf{GDP Per Capita USD}: 16,000 (2017), 2nd in LAC (PPP).
\item \textbf{Avg. annual real GDP growth 2011-2017}: 3.0 \%
\item \textbf{Poverty, less than \$5.5 (2011 PPP)}: 2011: 6\%; 2016: 3.7\%	
\end{itemize}
\begin{center}
	\includegraphics[width=60mm]{map.png}
	\label{map}
\end{center}
\end{frame}

\begin{frame}
\frametitle{UCT program: food card}
\begin{itemize}
\item \textbf{Managed by}: Ministry of Social Development.
\item \textbf{Amount}: Depends in a non-linear way in number of children. 47 USD (monthly) in 2019 for household with two kids. Double amount for poorest 30,000 households.
\item \textbf{Beginnings}: PANES households (target poorest 20\%) with children or pregnant women administratively enrolled in 2008 + 20,000 INDA households in 2009.
\item \textbf{Target}: Initially ``all households with children in extreme poverty''.		
\end{itemize}
\end{frame}

\begin{frame}
\frametitle{Targeting issues}
\begin{itemize}
\item \textbf{Recognized}: On October 2011 a report by the Ministry identified severe targeting issues.
\begin{itemize}
\item Households in extreme poverty were 10,000 in 2009 and beneficiaries were 85,000.
\item Not clear which is the target population.
\end{itemize}
\item \textbf{Steps taken in 2012}:
\begin{itemize}
\item Set target population to 60,000 poorest households ($\approx$ poorest 5\%).
\item Proxy means-tested approach to decide whether to grant or withdraw UCT benefits.
\item Increase household visits to (re) assess their status in the program.
\end{itemize}	
\end{itemize}
\end{frame}

\begin{frame}[label=nHouse]
\frametitle{Household visits}
\begin{center}
\includegraphics[width=80mm]{nHouse.pdf}
\label{nHouse}
\end{center}
{\small Who and how decided to visit which households?}
\begin{itemize}
	\item {\small``Requested'' visits: By phone or in one of the 38 regional offices.}
	\item {\small``Non-requested'' visits: 2011 Census, administrative records from other programs, other ad-hoc criteria. \hyperlink{MIDESCenso}{\beamergotobutton{Census map}}}
\end{itemize}
\end{frame}

\begin{frame}
\frametitle{Administrative data from the Ministry of Development}
\begin{center}
	\includegraphics[width=115mm]{summ.png}
	\label{summ}
\end{center}
\end{frame}

\begin{frame}
\frametitle{Questionnaire}
\begin{center}
\includegraphics[width=105mm]{mides.jpg}
\label{mides}
\end{center}
\end{frame}

\begin{frame}
\frametitle{Vulnerability Index}
\begin{itemize}
\item Construction: Estimated a probit in 2007 where $Y$ is a dummy equal to 1 when household belongs to the first income quintile (ECH 2006 sample: household below median income and with children).
\item Independent variables considered in the estimation are confidential.
\item Model run separately for Montevideo and the rest of the country.
\item Thresholds defined to capture target population.
\item Households never learn their Vulnerability Index.	
\end{itemize}
\end{frame}

\begin{frame}
\frametitle{Distribution of visits by Vulnerability Index (Montevideo)}
\begin{center}
	\includegraphics[width=110mm]{mdeodistrib.pdf}
	\label{mdeodistrib}
\end{center}
\end{frame}

\begin{frame}
\frametitle{Distribution of visits by Vulnerability Index (Rest of the country)}
\begin{center}
	\includegraphics[width=110mm]{intdistrib.pdf}
	\label{intdistrib}
\end{center}
\end{frame}

\begin{frame}
\frametitle{Empirical strategy}
\begin{itemize}
	\item FRDD with the Vulnerability Index as running variable and beneficiary status 1yr after the visit as the endogenous regressor.
	\begin{itemize}
		\item First stage:	
		\begin{align*}
		UCT_{i,t+12} & = \beta _0 + \beta_1\mathbf{1}[VI_{i,t}>0] + f(VI_{i,t}) + \epsilon_{i,t+12}
		\end{align*}
		\item Second stage:	
		\begin{align*}
		Y_{T} & = \beta _0 + \beta_1\hat{UCT}_{i,t+12} + f(VI_{i,t}) + \epsilon_{i,T}
		\end{align*}
	\end{itemize}
	\item DID:
	\begin{itemize}
		\item Gains: Control = ``0 to 0'', Treated = ``0 to 2''
		\item Losses: Control = ``2 to 2'', Treated = ``2 to 0''
	\end{itemize}
\end{itemize}
\end{frame}

\begin{frame}
\frametitle{Balance check: household income}
\begin{center}
	\includegraphics[width=110mm]{hogingtotsintransfOne.pdf} 
	\label{hogingtotsintransfOne}
\end{center}
\end{frame}

\begin{frame}
\frametitle{Balance check: food insecurity}
\begin{center}
	\includegraphics[width=110mm]{sinalimentos1.pdf} 
	\label{sinalimentos1}
\end{center}
\end{frame}

\begin{frame}
\frametitle{Balance check: schooling}
\begin{center}
	\includegraphics[width=110mm]{asisteEscuela1.pdf} 
	\label{asisteEscuela1}
\end{center}
\end{frame}

\begin{frame}
\frametitle{Balance check: computer at home}
\begin{center}
	\includegraphics[width=110mm]{tienecomputadorSi1.pdf} 
	\label{tienecomputadorSi1}
\end{center}
\end{frame}

\begin{frame}
\frametitle{McCrary test for selective sorting}
\begin{center}
	\includegraphics[width=75mm]{mccraryPrimTot.png} 
	\label{mccrary}
\end{center}
\begin{itemize}
	\item UCT first threshold t-stat: 0.16
\end{itemize}
\end{frame}

\begin{frame}
\frametitle{First stage for non-beneficiaries: 1 month after the visit}
\begin{center}
	\includegraphics[width=105mm]{int_noTus_tus1.pdf}
	\label{int_noTus_tus1}
\end{center}
\end{frame}

\begin{frame}
\frametitle{First stage for non-beneficiaries: 3 month after the visit}
\begin{center}
\includegraphics[width=105mm]{int_noTus_tus3.pdf}
\label{int_noTus_tus3}
\end{center}
\end{frame}

\begin{frame}
\frametitle{First stage for non-beneficiaries: 6 month after the visit}
\begin{center}
\includegraphics[width=105mm]{int_noTus_tus6.pdf}
\label{int_noTus_tus6}
\end{center}
\end{frame}

\begin{frame}
\frametitle{First stage for non-beneficiaries: 9 month after the visit}
\begin{center}
\includegraphics[width=105mm]{int_noTus_tus9.pdf}
\label{int_noTus_tus9}
\end{center}
\end{frame}

\begin{frame}
\frametitle{First stage for non-beneficiaries: 1 year after the visit}
\begin{center}
	\includegraphics[width=105mm]{int_noTus_tus12.pdf}
	\label{int_noTus_tus12}
\end{center}
\end{frame}

\begin{frame}
\frametitle{First stage for beneficiaries: 1 month after the visit}
\begin{center}
	\includegraphics[width=105mm]{int_si1Tus_tus1.pdf}
	\label{int_si1Tus_tus1}
\end{center}
\end{frame}

\begin{frame}
\frametitle{First stage for non-beneficiaries: 1 year after the visit}
\begin{center}
\includegraphics[width=105mm]{int_si1Tus_tus12.pdf}
\label{int_si1Tus_tus12.pdf}
\end{center}
\end{frame}

\begin{frame}
\frametitle{Outcomes}
\begin{itemize}	
	\item Formal labor supply
	\item Enrollment in another welfare program
	\item Complaints to the government
	\item Food insecurity
\end{itemize}
\end{frame}

\begin{frame}
\frametitle{Outcomes}
\begin{itemize}
	\item Formal labor supply
	\item {\color{gray}Enrollment in another welfare program}
	\item {\color{gray}Complaints to the government}
	\item {\color{gray}Food insecurity}
\end{itemize}
\end{frame}


\begin{frame}[shrink=5, label=masocupadoSIIAS36AllWC]
\frametitle{Formal labor supply}
\begin{center}
	\includegraphics[width=105mm]{masocupadoSIIAS36AllWC.pdf}
	\label{masocupadoSIIAS36AllWC}
\end{center}
\hyperlink{masocupadoSIIAS36AllSC}{\beamergotobutton{No controls}}
\hyperlink{masocupadoSIIAS36AllPlaSC}{\beamergotobutton{At t=0, no controls}}
\hyperlink{masocupadoSIIAS36AllPlaWC}{\beamergotobutton{At t=0, with controls}}
\end{frame}

\begin{frame}[shrink=20, label=FRDDLabor]
\frametitle{FRDD: Formal labor supply}
\medskip
\medskip
\medskip
\medskip
\begin{figure}
	\input{../Empirical_analysis/Analysis/Output/masocupadoSIIAS36.tex}
\end{figure}
\begin{itemize}
	\item Sample: Montevideo, ages (at the time of the visit) 18-60, visit took place in 2011 - 2015. Controls: $age, age^2$, female, year FE, formally employed at $t=0$.
\end{itemize}
\hyperlink{RDDrobustness}{\beamergotobutton{Robustness column 2}}
\end{frame}


\begin{frame}[label=flabor02]
\frametitle{Formal labor supply: from 0 to 2}
\begin{center}
	\includegraphics[width=105mm]{DID0ocupadosSIIASSuperGanar.pdf}
	\label{DID0ocupadosSIIASSuperGanar}
\end{center}

{\tiny Sample: Uruguay, ages (at the time of the visit) 18-60, visit took place in 2011 - 2015, not formally employed at $t=0$.} 
\hyperlink{flabor01}{\beamergotobutton{0 to 1}}
\end{frame}

\begin{frame}[label=flabor20]
\frametitle{Formal labor supply: from 2 to 0}
\begin{center}
	\includegraphics[width=105mm]{DID0ocupadosSIIASSuperPerder.pdf}
	\label{DID0ocupadosSIIASSuperPerder}
\end{center}
\hyperlink{flabor10}{\beamergotobutton{1 to 0}}
\end{frame}

\begin{frame}
\frametitle{Outcomes}
\begin{itemize}	
	\item {\color{gray}Formal labor supply}
	\item Enrollment in another welfare program
	\item {\color{gray}Complaints to the government}
	\item {\color{gray}Food insecurity}
\end{itemize}
\end{frame}

\begin{frame}
\frametitle{Welfare program for workers}
\begin{itemize}
	\item Family allowance for workers: approximately 180,000 children.
	\item Amount: USD 12-24 per child per month (distributed twice per year).
	\item \textbf{High take-up costs}: needs signature of the principal of the school, proof of marriage, child's ID and the ID of all members of the household, proof of birth, health records, agreement signed at a judicial center if child lives with only one parent.
	\item Beneficiaries:
	\begin{itemize}
		\item Aged $<$ 18 and in school.
		\item Guardian has to either (i) work in the (formal) private sector in one of these sectors: rural, industry and commerce, domestic service, construction; (ii) retired (same sectors); (iii) small rural producer.
		\item Household's formal wage income has to be lower than USD 1,493 per month.
	\end{itemize}
\end{itemize}
\end{frame}

\begin{frame}
\frametitle{Welfare program for workers}
\begin{figure}
	\includegraphics[width=105mm]{RDDbps_afam_ley_benef.pdf}
\end{figure}
\end{frame}

\begin{frame}[label=welfare02]
\frametitle{Welfare program for workers: from 0 to 2}
\begin{figure}
	\includegraphics[width=105mm]{DIDbpsGanar.pdf}
\end{figure}
\end{frame}

\begin{frame}[label=welfare20]
\frametitle{Welfare program for workers: from 2 to 0}
\begin{figure}
	\includegraphics[width=105mm]{DIDbpsPerder.pdf}
\end{figure}
\end{frame}

\begin{frame}
\frametitle{Outcomes}
\begin{itemize}
	\item {\color{gray}Formal labor supply}
	\item {\color{gray}Enrollment in another welfare program}
	\item Complaints to the government
	\item {\color{gray}Food insecurity}
\end{itemize}
\end{frame}

\begin{frame}
\frametitle{Complaints: demanding another visit}
\begin{center}
	\includegraphics[width=105mm]{DpedidoRevisitedOne.pdf}
	\label{DpedidoRevisitedOne}
\end{center}
\end{frame}

\begin{frame}
\frametitle{Complaints: demanding another visit}
\begin{figure}
	\input{../Empirical_analysis/Analysis/Output/DpedidoRevisitedOne1.tex}
\end{figure}
\end{frame}

\begin{frame}
\frametitle{Outcomes}
\begin{itemize}
	\item {\color{gray}Formal labor supply}
	\item {\color{gray}Enrollment in another welfare program}
	\item {\color{gray}Complaints to the government}
	\item Food insecurity
\end{itemize}
\end{frame}

\begin{frame}
\frametitle{No food because of lack of money in the previous 30 days}
\begin{center}
	\includegraphics[width=105mm]{sinalimentosTwo.pdf}
	\label{sinalimentosTwo}
\end{center}
\end{frame}

\begin{frame}
\frametitle{No food for minors}
\begin{center}
\includegraphics[width=105mm]{menornocomioTwo.pdf}
\label{menornocomioTwo}
\end{center}
\end{frame}

\begin{frame}
\frametitle{No food for adults}
\begin{center}
\includegraphics[width=105mm]{adultonocomioTwo.pdf}
\label{adultonocomioTwo}
\end{center}
\end{frame}

\begin{frame}
\frametitle{Concerns}
\begin{itemize}
\item Are differences due to differential miss-reporting rates on either side of the threshold?
\item {\color{gray}Are differences due to differential selection on who gets revisited on either side of the threshold?}
\end{itemize}
\end{frame}

\begin{frame}
\frametitle{Actual vs Reported CCT amount}
\begin{center}
\includegraphics[width=105mm]{mienteHogAFAMTwo.pdf}
\label{mienteHogAFAMTwo}
\end{center}
\end{frame}

\begin{frame}
\frametitle{Concerns}
\begin{itemize}
\item {\color{gray}Are differences due to differential miss-reporting rates on either side of the threshold?}
\item Are differences due to differential selection on who gets revisited on either side of the threshold?
\end{itemize}
\end{frame}

\begin{frame}
\frametitle{No food because of lack of money in the previous 30 days}
\begin{center}
\includegraphics[width=105mm]{sinalimentosOne.pdf}
\label{sinalimentosOne}
\end{center}
\end{frame}

\begin{frame}
\frametitle{No food for minors}
\begin{center}
\includegraphics[width=105mm]{menornocomioOne.pdf}
\label{menornocomioOne}
\end{center}
\end{frame}

\begin{frame}
\frametitle{No food for adults}
\begin{center}
\includegraphics[width=105mm]{adultonocomioOne.pdf}
\label{adultonocomioOne}
\end{center}
\end{frame}

\begin{frame}
\frametitle{Food insecurity: gaining vs losing a transfer}
\begin{figure}
\input{../Empirical_analysis/Analysis/Output/sinalimentosTwo1.tex}
\end{figure}
\end{frame}

\begin{frame}
\frametitle{Food insecurity for minors: gaining vs losing a transfer}
\begin{figure}
\input{../Empirical_analysis/Analysis/Output/menornocomioTwo1.tex}
\end{figure}
\end{frame}

\begin{frame}
\frametitle{Food insecurity for adults: gaining vs losing a transfer}
\begin{figure}
\input{../Empirical_analysis/Analysis/Output/adultonocomioTwo1.tex}
\end{figure}
\end{frame}

\begin{frame}
\frametitle{Robustness: ``Two-stage'' RDD}
\begin{itemize}
	\item Those that were getting UCT at the 1st visit could have different unobservables than those that were not getting UCT. Are losses vs gains differences due to these unobservables or due to differences in how people respond to losses vs gains?
	\item We perform a ``Two-stage'' RDD to control for unobservables differences between these two groups:
	\begin{itemize}
		\item Pick a bandwith and define the group of ``winners'' as those that were getting no UCT during their 1st visit and whose VI at the 1st visit was greater than the UCT threshold but within the bandwith. Analogous for ``losers''.
		\item Losing: Impact of being ``slightly'' below threshold at their 2nd visit for group of ``winners''.
		\item Gaining: Impact of being ``slightly'' above threshold at their 2nd visit for group of ``losers''.   
	\end{itemize}
\end{itemize}
\end{frame}

\begin{frame}
\frametitle{Conclusions/Summary of results}
\begin{itemize}
	\item \textbf{Labor supply}: Gaining the UCT has a negative impact on formal labor supply (-4pp, $\epsilon \approx -0.6$). Impact of losing is less robust.
	\item \textbf{Spillovers}: When the benefit is cut, total social spending does not decrease 1:1 as individuals sort themselves on other welfare programs (losing the UCT has a positive +2pp impact on enrolling in the ``welfare for workers'' program).
	\item \textbf{Complaints}: Losing the UCT increases the prob. of having a ``requested'' visit by 26pp. These ``complaints'' to the government are twice more elastic to losses than to gains.
	\item \textbf{Food insecurity}: Losing the UCT increases the prob. of being food insecure (self-reported) by 22pp. Elasticity of gains is slightly higher (in absolute value).
\end{itemize}
\end{frame}

\begin{frame}
\frametitle{Next steps}
\begin{itemize}
	\item \textbf{Other outcomes}: education, birth weight, spillovers to other welfare programs, {\color{gray}voting}.
	\item More robustness, peer effects.
	\item {\color{gray}Small experiment to study the mechanisms.}
	\item {\color{gray}Model the ``upper maximization problem'': what type of welfare systems/targeting mechanisms would different politicians design/use considering the behavioral responses seen here?}
	\item {\color{gray}Model the ``lower maximization problem''?}	
\end{itemize}
\end{frame}



\begin{frame}
\begin{center}
	{\Huge Questions?\par}
\end{center}
\end{frame}

\begin{frame}
\begin{center}
	{\Huge Thank you\par}
\end{center}
\end{frame}

\begin{frame}[label=flabor01]
\frametitle{Formal labor supply: from 0 to 1}
\begin{center}
	\includegraphics[width=105mm]{DID0ocupadosSIIASGanar.pdf}
	\label{DID0ocupadosSIIASGanar}
\end{center}
\hyperlink{flabor02}{\beamergotobutton{0 to 2}}
\end{frame}

\begin{frame}[label=flabor10]
\frametitle{Formal labor supply: from 1 to 0}
\begin{center}
\includegraphics[width=105mm]{DID0ocupadosSIIASPerder.pdf}
\label{DID0ocupadosSIIASPerder}
\end{center}
\hyperlink{flabor20}{\beamergotobutton{2 to 0}}
\end{frame}

\begin{frame}[shrink=5, label=masocupadoSIIAS36AllSC]
\frametitle{Formal labor supply (no controls)}
\begin{center}
	\includegraphics[width=105mm]{masocupadoSIIAS36AllSC.pdf}
	\label{masocupadoSIIAS36AllSC}
\end{center}
\hyperlink{masocupadoSIIAS36AllWC}{\beamergotobutton{Go back}}
\end{frame}

\begin{frame}[shrink=5, label=masocupadoSIIAS36AllPlaSC]
\frametitle{Formal labor supply at t=0 (no controls)}
\begin{center}
\includegraphics[width=105mm]{masocupadoSIIAS36AllPlaSC.pdf}
\label{masocupadoSIIAS36AllPlaSC}
\end{center}
\hyperlink{masocupadoSIIAS36AllWC}{\beamergotobutton{Go back}}
\end{frame}

\begin{frame}[shrink=5, label=masocupadoSIIAS36AllPlaWC]
\frametitle{Formal labor supply at t=0 (with controls)}
\begin{center}
\includegraphics[width=105mm]{masocupadoSIIAS36AllPlaWC.pdf}
\label{masocupadoSIIAS36AllPlaWC}
\end{center}
\hyperlink{masocupadoSIIAS36AllWC}{\beamergotobutton{Go back}}
\end{frame}

\begin{frame}[label=RDDrobustness]
\frametitle{Robustness}
\begin{figure}
	\includegraphics[width=105mm]{RDDrobustness.pdf}
	\label{RDDrobustness}
\end{figure}
\hyperlink{FRDDLabor}{\beamergotobutton{Go back}}
\end{frame}

\begin{frame}[label=MIDESCenso]
\frametitle{Using the 2011 Census to define visits}
\begin{center}
	\includegraphics[width=100mm]{C:/Alejandro/Research/MIDES/Empirical_analysis/Other/Pictures/MIDESCenso.png}
	\label{MIDESCenso}
\end{center}
\hyperlink{nHouse}{\beamergotobutton{Go back}}
\end{frame}

\end{document}	